\documentclass[a4paper,12pt]{article}
\usepackage[utf8]{inputenc}
\usepackage[a4paper, top=0.2in, bottom=0.2in, left=1in, right=1in]{geometry}
\usepackage{csquotes}
\usepackage[style=apa, backend=biber]{biblatex}
\DeclareLanguageMapping{american}{american-apa}
\addbibresource{ref.bib} % Your bibliography file
\title{\textbf{Statement of Purpose}}
\date{}

\begin{document}

\maketitle
\noindent
\textbf{Group Member(s):} \\
{\small Rahul Biswas\,[Class Roll 10], Arijit Mondal\,[Class Roll 27],
Intekhab Hussain[Class Roll 36].}\\

\noindent
\textbf{Proposed Domain:} 
Bioinformatics / Computational Biology.

\vspace{12pt}
\noindent
\textbf{Broad Objective:} \\
This objective aims to understand the complex interactions between genes in living systems by developing computational methods to analyze and interpret high-throughput genomic data. The goal is to create accurate models of gene regulatory networks that can explain and predict biological processes at a systemic level.

\vspace{12pt}
\noindent
\textbf{Short Description:} \\
The genomic revolution seeks to understand the genetic basis of phenotypic traits in organisms, revealing how genes and proteins interact to form complex living systems. Advances in genome-wide gene expression technologies bring us closer to mapping gene networks—interactions among genes within a living system. These networks are crucial for linking biology with computational science through both qualitative and quantitative modeling.

By modeling and simulating gene networks, we can predict biological behavior, understand gene regulation, and explore connections between gene regulation and phenotype. This knowledge is essential for applications in \textbf{personalized medicine}, \textbf{drug targeting}, and \textbf{evolutionary studies}. Though our understanding of gene networks is still developing, recent biotechnological progress has greatly enhanced our ability to accurately model these networks, providing deeper insights into the fundamental processes of life.(\cite{aluru2005handbook})

\vspace{12pt}
\noindent
\textbf{Special Mention:} \\
We began our work under the guidance of \textbf{Professor Rajat Kumar Pal} at the beginning of the second semester. Several significant projects have already been completed in this area, including \enquote{\textit{\textbf{Modified Half-System Based Method for Reverse Engineering of Gene Regulatory Networks}}}(\cite{Half-system}) and \enquote{\textit{\textbf{Inference of Genetic Regulatory Networks with Recurrent Neural Network Models Using Particle Swarm Optimization.}}}(\cite{RNN-model})
{\small
\printbibliography
}

\end{document}
